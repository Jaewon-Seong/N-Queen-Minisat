\documentclass{article}
\usepackage[utf8]{inputenc}
\usepackage{graphicx}

\title{LCS Project}
\author{Solving N-Queens Problem using SAT Solver
}
\date{07 November 2022}

\begin{document}

\maketitle

\section{ Satisfiability and SAT Solver}
\\
\text{Subject : CS 5384 Logic for Computer Scientist
Professor : Yuanlin Zhang\\ 
\\ \\
Member :\\ 
         Jaewon Seong,\\
         Teja Sai Srinivas Khajjayam,\\
         Vanaja Saya\\
} \\
\\
\\
\\ \\ 
\\ \\
\section{ Objective}
Download a SAT solver and how to install it and use it
\\ \\
Encode the n-queens problem into the DIMACS CNF format
      and solve it by the SAT
\\ \\
\\ \\ \\ \\ 
\\ \\ \\ \\
\section{Outline} 


1)How to install a SAT solver
\\ \\
2)How to set the program
\\ \\
3)Satisfiability
\\ \\
4)SAT Solver (MiniSAT)
\\ \\
5)Solving N-Queens using a SAT Solver
\\ \\
\section{    How to install a SAT solver
}
\includegraphics[width=1.3\textwidth]{01.png} \\
\\
\includegraphics[width=1.3\textwidth]{02.png}\\
\\
\section{How to set a SAT solver}
\includegraphics[width=1.3\textwidth]{03.png} \\
\\
\includegraphics[width=1.3\textwidth]{04.png}\\
\\
\section{Satisfiability}
1)A Boolean expression is satisfiable if there is an assignment to the variables that makes it true
\\ \\
2)Checking to see if a formula S is satisfiable can be done by searching a truth table for a true entry

\section{SAT Solver (MiniSAT)}
Input expected in CNF
\\ \\
Use DIMACS CNF format
\\ \\
   The input file starts with comments (each line start with c)
\\ \\
      The number of variables and the number of clauses defined by the line p cnf \#variables \#clauses
\\ \\
      Each of the next lines specifies a clause: a positive literal is denoted by the corresponding number, and a negative literal is denoted by the corresponding negative number
\\ \\
    Each of the next lines specifies a clause: a positive literal is denoted by the corresponding number, and a negative literal is denoted by the corresponding negative number
\\ \\
Use MiniSAT (minisat.se)
\\ \\
\section{SAT Solver (MiniSAT Example)}
DIMACS 	format
\\ \\
(c = comment, ‘p cnf’ = SAT problem in CNF)
\\ \\
C SAT problem in CNF with 2 variables and 2 clauses
\\ \\
p cnf 2 2
\\ \\
-1 -2 0
\\ \\
-1 2 0
\\ \\
\section{SAT Solver (MiniSAT Example Input)}
\includegraphics[width=1.3\textwidth]{05.png} \\
\\ \\
\section{SAT Solver (MiniSAT Example Output)}
\includegraphics[width=1.3\textwidth]{06.png}\\
\\
\section{Pseudo code for \\
  Solving N-Queens using a SAT Solver 
}
\includegraphics[width=1.3\textwidth]{13.PNG} \\
\\
1)We will take the Value of N from the user.\\ \\
2)We will next Open the file with "w" Permissions.\\ \\
3)We will append all the elements in the same row and column.\\ \\
4)We will append the elements in the diagonal down from 0 to N .\\ \\
5)We will append the elements in the diagonal down from N to 0 .\\ \\
6)We will append the elements in the diagonal UP from 1 to N .\\ \\
7)We will append the elements in the diagonal UP from N to 1 .\\ \\
8)We will next write the line with "p cnf 'Number of 'Number of values in t\\  Board' 'No.of Clauses' " at the starting of the file.\\ \\
9)We will send that file as input to Sat Solver to solve\\ \\
10) We will Store the output cnf file.\\ \\
11)We now print the generated cnf file in the 2-D Array for the better\\ understanding. \\ \\
12)The places of the queens where it needs to be placed is shown.\\ \\

\section{Solving N-Queens using a SAT Solver (Input)}
\includegraphics[width=1.3\textwidth]{07.png} \\
\\
\includegraphics[width=1.3\textwidth]{08.png}\\
\\
\includegraphics[width=1.3\textwidth]{09.png} \\
\\
\includegraphics[width=1.3\textwidth]{10.png}\\
\\
\\\includegraphics[width=1.3\textwidth]{11.png} \\
\\

\section{Solving N-Queens using a SAT Solver (Output)}
\includegraphics[width=1.3\textwidth]{12.png}\\
\\
\end{document}
